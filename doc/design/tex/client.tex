\chapter{client}
\label{chapter:client}

The client normally forwards the input on the console to the server. Every
line is a command. Lines that start with a slash `/' are handled by the
client itself. Thus it is not possible to send commands or lines to the
server that start with a slash.

In order to send commands using multiple lines see
\cref{section:client:begin} and \cref{section:client:end}.

The rest of the chapter lists the commands in alphabetic order.



\section{Connect}
\label{section:client:connect}

Connects to a server.

\indent connect SERVER[:SERVICE]\\

After the connection with the server is established the protocol is switched
to \nameref{protocol:basic}.

At the moment the connection is always using IPv4, IPv6 support is planned
for the future.


xxxx (should be Koma thingy)
\begin{description}
\item[SERVER]
	The hostname or IP address of the server to connect to.

\item[SERVICE]
	The service name or the port number to connect to. If omitted the
	default port, $2048$, is used.

\end{description}

The following errors are expected:xxxx


\section{Begin}
\label{section:client:begin}

Begins a multi-line command.

\indent begin

Every line typed after this command is buffered, including the end of line
character,  until the \nameref{section:client:end} command is entered. Then
the entire buffer, minus the last end of line character is send to the
server as one command. For example:

\begin{alltt}
/begin
message This is the first line in the message.
This is the second line in the message.
This is the third line in the message.
/end
\end{alltt}

This sends a message containing three lines.

After this command only the \nameref{section:client:end} command is allowed
all other commands result in undefined behaviour.



EBUSY if already active


\section{end}
\label{section:client:end}

Ends a multi-line command.

\indent end

For more information see \cref{section:client:begin}.

EPROTOCOL if not in multi-line mode


\section{Log}
\label{section:client:log}

Sets the filename to log to.

\indent log FILENAME

The server logs all communication with the server in a logfile. The default
name is `client.log'. All log entries are appended to the file.

Upon error the old log file is still used.

\begin{description}
\item[FILENAME]
	The name of the file to log to. When the file does not exists it is
	created, but the parent directory must exist.
\end{description}

Errors:
\begin{description}
\item[EBUSY]
	The file cannot be opened since it is opened by another process.

\item[EPERM]
	The file cannot be opend due to lack of permissions.

\item[EXISTS]
	The item exists and is a directory.

\end{description}


%%%\begin{protocol}
%%%\label{protocol}

Changes the communication protocol.

\indent protocol PROTOCOL

xxxx \nameref{type:protocol} for the type?

This command allows to switch to every protocol however it is recommended to
%%%only switch to \nameref{protocol:basic} and \protocol{compressed}% add the
%protocol command
The other protocols should work but are only intended to be used for
debugging.

\begin{description}
\item[PROTOCOL]
	The new protocol to use.
\end{description}

