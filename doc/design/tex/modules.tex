\chapter{modules}
\label{chapter:modules}

This chapter contains all modules. When a module is large it might have its
own dedicated chapter as well.


\section{Commmunication}
\label{section:module:communication}

The communication module contains several classes used for communication.
These classes are gathered in the main classes \command|tsocket| and
\command|tfile|. The other classes are helpers and implementation details,
but since the memberfunctions of these classes are important to know they
are not in a \command|detail| namespace.

The \command|tsocket| class contains the implementation for a tcp socket.

The \command|tfile| class contains the implementation for reading from
certain file descriptor. This code probably won't work on Windows.

The generic design of these classes is not cast in stone yet. It has mainly
be written as proof-of-concept and still needs to be polished.

The classes can be set in sync and async mode. Regardless of the mode
everything goes asynchrounously.


\section{Lobby}
\label{section:module:lobby}

This module contains the lobby for the server. Its goal is to manage all
connected users.


\section{Logging}
\label{logging}

This module has several loggers:
\begin{description}
\item[simple]
	The simple logger has no logging levels and logs everything. It can also
	be used to write to log files, that are accessed from multiple threads.

\item[basic]
	The basic logger also enables log levels.

\item[module]
	The module logger also adds modules, where every module has its own
	log level.

\end{description}
