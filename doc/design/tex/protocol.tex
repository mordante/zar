\chapter{Protocol}
\label{chapter:protocol}

The protocol described here will first describe some basic features of the
protocol. Then the types are explained, followed by some information
regarding the modes. The details of the modes are described in other
chapters.

\section{Generic}
\label{section:protocol:generic}

The protocol uses the UTF-8 character encoding and an end of line is the
UNIX end of line. The communication can both be initiated by the server or
the client. For example client A sends a message to the server requesting
to send `hello world' to all players in the lobby. At that moment only
player A and B are in the lobby. The following messages will be send:
\begin{enumerate}
\item
	Player A sends the request to the server, this is an action.
\item
	The server sends the `hello world' to player B, this is an action.
\item
	Player B acknowledges the response to the server, this is a reply.
\item
	The server responds to player A, this is a reply.
\end{enumerate}

In this example the server wait with the response until it knows all players
have acknowledged the message. The exact details of the message depend on
the type.

XXX strip connection message on the server, but explain there might be a
motd which expects no response.

\section{Types}
\label{section:protocol:types}




\subsection{Line}
\label{protocol:line}

Only used on the client and basically the same as the
\nameref{protocol:telnet}, but only uses the UNIX end of line as line
terminator.


\subsection{Telnet}
\label{protocol:telnet}

This is the default protocol when starting a connection to the server. The
protocol is not the default since it is the best protocol to use, but
because it is easy to use when debugging with a telnet client.

The protocol sends a string of text ending with the telnet end of line.

The reply consists of a string possibly containing multiple lines
(can it have end of lines in the message???)

The normal client software will switch to the \nameref{protocol:basic} as
soon as possible. The text based client needs to switch manually, but this
client also is not the greatest of clients to use.

\subsection{Basic}
\label{protocol:basic}

The basic protocol is also text based and contains the following fields:
\begin{description}
\item[message length]
	A \mbox{$32$-bit} value in network byte order. This value contains the
	size of the rest of the message, including the other header parts.

\item[message type]
	A one character value determining the type of the message. The possible
	values are:
	\begin{description}
	\item[A] The message is an action.
	\item[R] The message is a reply.
	\end{description}

\item[message id]
	A \mbox{$32$-bit} value in network byte order. This id is used to map an
	\emph{action} with a \emph{reply}. The value is determined by the sender
	of the action.

\item[message contents]
	The actual contents of the message. All characters are allowed in the
	message and none are escaped\footnote{Due to the length prefix there is
	no need for escaping.}. The contents of the message depend on the
	message type:
	\begin{description}
	\item[A]
		The message contains an action. An action is a single word
		containing character from the following set (using the C locale)
		[a-Z0-9\_]. If the command has a parameter there will be \emph{one}
		space after the command, then its parameter. The server is a state
		machine, thus one parameter will suffice. E.g.
		\begin{itemize}% koma special list
		\item[set filetype=jpeg]
			Tells the server next file will be a jpeg file,
			when loading multiple jpeg files, it only needs to be set once.
			The next comm
		\item[set filename=portrait]
			Tells the server to store the file under the name portrait on
			the server.
		\item file JPEG\ldots
			Contains the actual contents of the file. The client should
			facilitate a way to load the file. For example, on the client
			one issues \command|/file no_elf.jpg|.
		\end{itemize}
	\item[B]
		The message contains a reply. A reply contains of a single word
		stating the result of the action. When the result is \command|OK|
		the action was a success, all other values are an error. If there is
		more to explain the result is followed by an end of line followed by
		the exact result. For example the command
		\mbox{\command|get_file portrait|} will return
		\mbox{\texttt{OK\\nJPEG\ldots}}. Not every command will have a
		result, but list of errors
	\end{description}


\end{description}


\subsection{Compressed}
\label{protocol:compressed}

This protocol is basically the same as \nameref{protocol:basic} but
compresses the transmitted data. The exact method of compression is not yet
determined. (Thus the protocol is not yet implemented.)

\section{Command mode}
\section{section:protocol:command\_mode}

command mode used as mode on server
command\_version used as version for the commands
protocol\_version used as version of the protocol

