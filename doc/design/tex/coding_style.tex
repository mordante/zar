\chapter{Coding style}
\label{chapter:coding_style}

The programmes are writen in \mbox{C++} based on the \mbox{C++$11$}
standard. The number of compilers implementing this standard are
rare\footnote{Written in november $2011$.}, the only compiler the programmes
are tested with are pre-releases of \mbox{gcc-4.7}. Even this compiler has
no full \mbox{C++$11$} support, but its support is rather good.


\section{Separation}

The code is separated in libraries and modules. There is no real difference
between a library and a module. In general a module is a large library. Both
libraries and modules can depend on other libraries and modules in the
packages. All code in the libraries is in the \command|lib| namespace, all
code in a modules is in a namespace named after the module. Code which
contains implementation details and should not be used outside the library
or module is nesteed in the \command|detail| namespace.

Files of a library are in a directory named after the library in the
libraries subdirectory. Modules in a directory in named after the module in
the modules subdirectory. Libraries most of the time have no other
subdirectories and a limited number of files. Modules are expected to have a
larger number of files and subdirectories. If a file only contains
implementation details it should be in a detail subdirectory.


Filenames have several extensions:
\begin{description}
\item[hpp]\label{file:hpp}
	Header files, they need to have an include guard and an associated
	\nameref{file:cpp} file.

\item[cpp]\label{file:cpp}
	Contains the implementation of its associated \nameref{file:hpp} file.

\item[tpp]
	These files are like \nameref{file:hpp} but has no associated
	\nameref{file:cpp} file. The file only contains templated code. This
	does not mean \nameref{file:hpp} files contain no templates.

\end{description}

A single file should normally only implements a single class.


\section{Documentation}

All code is documented using Doxygen. This documentation is meant as
reference documentation. The larger picture is described in this document.


\section{Coding style}

This section contains the coding style. The code is using \mbox{C++$11$} its
features are preferred to be used over \mbox{C++$98$} style features. For
example, a non-copyable class that can be moved, can be stored directly in a
standard container, no need to store a pointer.

\begin{description}
\item{Constructors, destructors and assignment}
	All these items need to be defined for every class. \mbox{C++$11$}
	allows \command|deleted| and \command|defaulted| versions of these
	functions, use this feature.

\item{Assign constructor values in the header}
	\mbox{C++$11$} allows to assign a constructor value in the header, use
	this for values that are assigned the same value for all constructors.

\item{Pointer and reference}
	If a parameter is a pointer its value may be a \command|nullptr| when a
	reference not.

\item{Exceptions}
	All code should be exception safe and use exceptions to indicate
	errors.

\item{No compilation warnings}
	By default the compiler settings are strict, make sure committed code
	does not trigger warnings. (Obviously when using a different compiler
	new warnings might be triggered.) On occation warnings might be disabled
	for a part of a tranistion unit, when the warning is triggered by
	external code.

\item{Document all code}
	The code should contain Doxygen comment. This comment is meant as
	reference documentation. The larger picture should be documented in this
	document.

\item{In rome do as the romans do}
	When starting to hack on the source code, emulate its style.

\end{description}
