\chapter{types}
\label{chapter:types}


pint, positive int
nint, negative int`:w

\section{Intergral}
\label{section:types:integral}

The integral types are all numeric types. The section shows each type from
the smallest to the largest.

\section{int8}
\label{type:int8}

\begin{description}
\item[int8]\label{type:int8}
	An \mbox{$8$-byte} integer value, whose range is [-128,127].

\item[uint8]\label{type:uint8}
	An \mbox{$8$-byte} unsigned integer value, whose range is [0,255].

\item[pint8]\label{type:pint8}
	An \mbox{$8$-byte} positive integer value, whose range is [1,255].

\item[nint8]\label{type:nint8}
	An \mbox{$8$-byte} negative integer value, whose range is [-1,-255].

% ....

% pint maps to the largest pintxx, at the moment 32

\item[float]\label{float}
	A floating point value, mapping to the C-type \command|double| its exact
	range and resolution depend on the hardware platform.

\end{description}


\section{Enumerate}
\label{type:enumerate}

This section contains serveral enumerate type. They are between the integral
type and string type, since internally they are a number, but external a
string, even more as the normal numbers.


\section{String}
label{section:type:string}

Contains all string types.


\subsection{string}
\label{type:string}

Contains a pure string.


\subsection{filename}
\label{type:filename}

Contains the name of a file. Filenames starting with a leading / are
absolute. Otherwise the filename is relative. Whether or not the file needs
to exists depends on the platform.

On the windows platform an absolute name normally starts with `driveletter
colon'. Here is still gets the leading slash. For example
\command|C:\my_file| is written as \command|/C:\my_file|.

\section{Misc}



\subsection{Dice}
\label{type:dice}

The dice type is a special type and is formatted like:

\indent [THROWS]dEYES[(-|+)MODIFIER]

\begin{description}
\item[THROWS]
	A \type{pint} determining the number of throws that are required.

\item[EYES]
	A \type{pint} determining the number of eyes on the dice. Every throw
	returns a value in the range [1, EYES].

\item[MODIFIER]
	a \type{uint}\footnote{positive, negative values are possible with the
	minus sign.} determining the final adjustment of the throw.

\end{description}

Note when using large values for the parameters the results might be odd.
The result of a throw is an \type{int} and has wrapping behaviour.

