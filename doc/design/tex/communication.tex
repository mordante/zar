\part{Communication}
\label{part:communication}

This part describes the communication protocol. It starts with the basic
protocol, followed a chapter of data types known to the protocol. The
commands available depend on the operation mode, first the operations
available in all modes are described, followed by the commands per mode.
The part ends with the commands in the client.


%---
%
%\chapter{Communication}
%\label{chapter:communication}
%
%This chapter first describes the communication protocols between the client and
%server. Then it explains the different modi in the communication and the
%available commands per mode.
%
%
%\subsection{Protocol}
%\label{section:communication:protocol}
%
%The communication specifies four different protocols, of which not all are
%implemented. The default protocol is telnet.
%
%\begin{description}
%\item[line]
%	The line based protocol ends every command with a newline character.
%	This protocol is not used on the network, but is used for the client.
%
%\item[telnet]
%	This protocol is also line based, but has a carriage return and linefeed
%	terminator. This is the default for the telnet programme. This protocol
%	has the disadvantage that it is not possible to use multiple lines in a
%	single command. It is the default since it makes it easier to debug and
%	test with the telnet programme.
%
%\item[length\_prefixed]
%	This protocol has 4 bytes (network order) containing the length of the
%	rest of the message. Then number of bytes need to be read.
%
%\item[gzipped]
%	Like the length\_prefixed, but the stream is compressed with gzip. It is
%	not yet determined whether the entire stream is gzipped or only the
%	message contents.
%\end{description}
%
%
%\subsection{Modi}
%\label{section:communication:modi}
%
%When connection to the server depending on the mode several commands are or
%are not available.
%
%
%\subsection{Generic}
%
%Always available:
%
%\minisec{set}
%The set command is always issued like
%
%set variable=value
%
%No white space is allowed around the assignment operator. The variables
%available depend on the active mode.
%
%
%
%
%\subsection{Connected}
%
%This mode is active directly after connecting to the server.
%
%
%\minisec{login}
%
%login ID
%
%The ID to use on the server. After issuing this command the mode changed to
%lobby.
%
%
%\subsection{Lobby}
%
%After logging in after connecting or after leaving a game this mode will be
%active.
%
%-------
%Set
%
%protocol=PROTOCOL
%
%PROTOCOL is one of \nameref{protocol:line}, \nameref{protocol:telnet},
%\nameref{protocol:base}, \nameref{protocol:compressed}.
%
%sets the appropriate communication protocol. The server response will
%already use the new protocol.
%
%Available modi \AllModi
%
%
%-----
%
%Maybe make this entire thing a part...
